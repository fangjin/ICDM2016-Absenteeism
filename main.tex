%% bare_conf.tex
%% V1.4b
%% 2015/08/26
%% by Michael Shell
%% See:
%% http://www.michaelshell.org/
%% for current contact information.
%%
%% This is a skeleton file demonstrating the use of IEEEtran.cls
%% (requires IEEEtran.cls version 1.8b or later) with an IEEE
%% conference paper.
%%
%% Support sites:
%% http://www.michaelshell.org/tex/ieeetran/
%% http://www.ctan.org/pkg/ieeetran
%% and
%% http://www.ieee.org/

%%*************************************************************************
%% Legal Notice:
%% This code is offered as-is without any warranty either expressed or
%% implied; without even the implied warranty of MERCHANTABILITY or
%% FITNESS FOR A PARTICULAR PURPOSE!
%% User assumes all risk.
%% In no event shall the IEEE or any contributor to this code be liable for
%% any damages or losses, including, but not limited to, incidental,
%% consequential, or any other damages, resulting from the use or misuse
%% of any information contained here.
%%
%% All comments are the opinions of their respective authors and are not
%% necessarily endorsed by the IEEE.
%%
%% This work is distributed under the LaTeX Project Public License (LPPL)
%% ( http://www.latex-project.org/ ) version 1.3, and may be freely used,
%% distributed and modified. A copy of the LPPL, version 1.3, is included
%% in the base LaTeX documentation of all distributions of LaTeX released
%% 2003/12/01 or later.
%% Retain all contribution notices and credits.
%% ** Modified files should be clearly indicated as such, including  **
%% ** renaming them and changing author support contact information. **
%%*************************************************************************


% *** Authors should verify (and, if needed, correct) their LaTeX system  ***
% *** with the testflow diagnostic prior to trusting their LaTeX platform ***
% *** with production work. The IEEE's font choices and paper sizes can   ***
% *** trigger bugs that do not appear when using other class files.       ***                          ***
% The testflow support page is at:
% http://www.michaelshell.org/tex/testflow/


\documentclass[conference]{IEEEtran}
\usepackage{subfigure}
\usepackage{xcolor}
\usepackage{url}
\usepackage{mathrsfs}
\usepackage{amsfonts}
\usepackage[normalem]{ulem}
\usepackage{array}
\usepackage{amsmath}
%\usepackage{underscore}
\usepackage{graphics}
\usepackage{graphicx}
\usepackage{algorithm}
\usepackage{algorithmic}
\usepackage{caption}
\newcommand{\argmin}{\arg\!\min}
\newcommand{\argmax}{\arg\!\max}


\usepackage[utf8]{inputenc}
\usepackage[english]{babel}
\newtheorem{theorem}{Theorem}[section]
\newtheorem{corollary}{Corollary}[theorem]
\newtheorem{lemma}[theorem]{Lemma}



% *** GRAPHICS RELATED PACKAGES ***
%
\ifCLASSINFOpdf
  % \usepackage[pdftex]{graphicx}
  % declare the path(s) where your graphic files are
  % \graphicspath{{../pdf/}{../jpeg/}}
  % and their extensions so you won't have to specify these with
  % every instance of \includegraphics
  % \DeclareGraphicsExtensions{.pdf,.jpeg,.png}
\else
  % or other class option (dvipsone, dvipdf, if not using dvips). graphicx
  % will default to the driver specified in the system graphics.cfg if no
  % driver is specified.
  % \usepackage[dvips]{graphicx}
  % declare the path(s) where your graphic files are
  % \graphicspath{{../eps/}}
  % and their extensions so you won't have to specify these with
  % every instance of \includegraphics
  % \DeclareGraphicsExtensions{.eps}
\fi


% correct bad hyphenation here
\hyphenation{op-tical net-works semi-conduc-tor}


\begin{document}
%
% paper title
% Titles are generally capitalized except for words such as a, an, and, as,
% at, but, by, for, in, nor, of, on, or, the, to and up, which are usually
% not capitalized unless they are the first or last word of the title.
% Linebreaks \\ can be used within to get better formatting as desired.
% Do not put math or special symbols in the title.
\title{Absenteeism Detection in Social Media}

%``The Lull before the Storm": Event Detection from Group Anomaly

%\title{Group Anomaly Detection using Graph Wavelet}
% author names and affiliations
% use a multiple column layout for up to three different
% affiliations

%\author{Fang Jin\footnotemark[1], Feng Chen\footnotemark[2], Rupinder Paul Khandpur\footnotemark[1], Chang-Tien Lu\footnotemark[1], Naren Ramakrishnan\footnotemark[1]\\[2mm]
%		\affaddr\footnotemark[1] {Discovery Analytics Center, Department of Computer Science, Virginia Tech.}\\
%\affaddr\footnotemark[2] {Department of Computer Science, University at Albany, SUNY}\\[0.5mm]
%        \footnotemark[1] \email {\{jfang8, rupen, ctlu, naren\}@cs.vt.edu }, \footnotemark[2] \email {fchen5@albany.edu}
%}
%\author{\IEEEauthorblockN{Fang Jin}
%\IEEEauthorblockA{Discovery Analytics Center\\
%Department of Computer Science\\
%Virginia Tech\\
%Email: jfang8@cs.vt.edu}
%\and
%\IEEEauthorblockN{Feng Chen and \\ Rupinder Paul Khandpur}
%\IEEEauthorblockA{Department of Computer Science\\
%University at Albany, SUNY\\
%Virginia Tech\\
%Email: fchen5@albany.edu, rupen@cs.vt.edu}
%\and
%\IEEEauthorblockN{Chang-Tien Lu and\\ Naren Ramakrishnanu}
%\IEEEauthorblockA{Discovery Analytics Center\\
%Department of Computer Science\\
%Virginia Tech\\
%Email: ctlu@vt.edu, naren@cs.vt.edu}
%}



% make the title area
\maketitle

% As a general rule, do not put math, special symbols or citations
% in the abstract
\begin{abstract}
Event detection in online social media has primarily focused on identifying
abnormal spikes, or bursts, in activity. However, disruptive events such as socio-economic disasters, civil unrest, and even power outages, often involve abnormal troughs or lack of activity,
leading to absenteeism.
We present the first study, to our knowledge, that models absenteeism and uses detected absenteeism
instances as a basis for event detection in location-based social networks (LBSN) such as Twitter. The proposed framework addresses the challenges of (i) early detection of absenteeism, (ii) identifying the locus of the absenteeism, and (iii) identifying groups or communities underlying the absenteeism. Our approach uses the formalism of graph wavelets to represent the spatiotemporal structure of
user activity in a LSBN. This formalism facilitates multiscale analysis, enabling us to detect anomalous behavior at different graph resolutions, which in turn allows the identification of event locations
and underlying groups.
The effectiveness of our approach is evaluated using Twitter activity related to civil unrest
events in Latin America.
\end{abstract}

% no keywords
% For peer review papers, you can put extra information on the cover
% page as needed:
% \ifCLASSOPTIONpeerreview
% \begin{center} \bfseries EDICS Category: 3-BBND \end{center}
% \fi
%
% For peerreview papers, this IEEEtran command inserts a page break and
% creates the second title. It will be ignored for other modes.
\IEEEpeerreviewmaketitle


\section{Introduction}
\label{sec:introduction}
Social microblogs such as Twitter and Weibo are experiencing explosive growth, with billions of users globally sharing their daily status updates online. For example, as of March 31, 2016 Twitter had more than 310 million average monthly active users (78\% of whom were using mobile devices) and were anticipating this to continue to grow by as much as 25\% per year\footnote{http://www.statista.com/statistics/282087/number-of-monthly-active-twitter-users/}.
Various studies have shown that Twitter is a viable `social sensor', and thus holds great promise for detecting and forecasting significant societal events ~\cite{bugel2013multilingual,sakaki2010earthquake}.
In recent years, a significant body of research~\cite{aggarwal2012event,hong2012discovering,lappas2009burstiness,lappas2012spatiotemporal,sakaki2010earthquake,sayyadi2009event,watanabe2011jasmine,weng2011event,yin2011geographical} has focused on modeling bursts and increases of user activity in social media.

However, real world events are not only correlated with burst signals, but can also lead to unusually low levels of activity in social networks. An example of this phenomenon
is shown in Figure 1, where a protest in the city of Natal, Brazil, began at 5:00 PM (local time) at the Museum of the Republic, with people gradually joining the demonstration. %\footnote{http://www.jb.com.br/pais/noticias/2013/06/17/manifestantes-invadem-cobertura-do-congresso-nacional-em-brasilia/}.
On Twitter, there was an uncharacteristic lull in activity or {\it group absenteeism} behavior in the area for the two hours from 6:00 PM - 8:00 PM that day. Another example comes from December 24, 2013, when southern Brazil experienced widespread flash floods. According to news sources, more than 50,000 people were forced to flee their homes in Minas Gerais and Espirito Santo, in the southern states of Brazil. Immediately following the floods, Twitter activity in this region dropped by 51\%, reaching its lowest point later that evening.
%Other examples of \textit{group absenteeism} that we observed from Latin American Twitter activity include bus strikes in Brazil on May 21, 2014, the Iquique earthquake in Chile on April 1, 2014, and a major power supply disruption in Argentina on December 30, 2013.

\begin{figure}[t]
\centering
\includegraphics[height=1.1in]{figures/Natal_example1.png}
\caption{Detected group absenteeism in Natal, Brazil beginning at 6:00 PM on June 17, 2013. This absenteeism event coincides with a large protest that happened in the region.}
\label{fig:natal-protest}
\end{figure}

Developing a better appreciation of this phenomenon of unusually calm behavior online holds enormous potential for understanding localized, disruptive, societal events. In this paper we focus on
absenteeism as a key phenomenon of interest and develop novel
group anomaly detection algorithms for this purpose.
An absenteeism event in a social network can be defined as an event which is characterized by a significant lull in activity such as a sudden, sharp decrease of Twitter volume within a short period of time (and which may precede a major burst in activity as people react to the event). This paper presents the first study to systematically investigate group anomaly in location-based social networks (LBSNs). To appropriately incorporate absenteeism concepts into our detection approach, we must first address the following questions:

\begin{itemize}
\item How can we define/adapt anomaly detection algorithms to capture not just bursty situations
but also those that involve absenteeism?
\item At what scale should we model the absenteeism activity and how can we isolate the locus of
interest?
\item What is the most efficient way to select abnormal groups that are spatially and temporally localized?
\item How do we model an absenteeism signal for event detection? Even though we have clear examples of real world events that explain the observed absenteeism, not all absenteeism occurrences will be associated with underlying events and thus we must be able to differentiate between absenteeism and merely
noisy signals for successful event detection.
\end{itemize}

A graph wavelet approach offers several outstanding advantages to study
studying the above questions, including scalability, localization, low computational complexity, and
compactness in defining groups. In this formalism, the data objects are embedded in a general graph as vertices. By employing wavelet transforms on the graph, we can construct a wavelet function with a graph structure. We propose the use of a graph anomaly index that depends on the graph structure in conjunction with an absenteeism score vector in order to define whether a graph is abnormal. When a graph is deemed to be exhibiting abnormal behavior, we can calculate its wavelet coefficient to identify the central node and its coverage area. This approach will enable us to select abnormal groups at different scales. Such group anomaly detection methods are varied and proven to be effective in detecting events such as protest marches and natural disasters.

%As for another abnormal scenarios such as natural disaster, we propose a two-pass group anomaly detection method that first detects absenteeism, and then checks if there is a subsequent burst in activity within a specific time period. By comparing correlations between the wavelet coefficients of both of these groups, we are able to capture a possible real world event.

Our contributions are thus:
\begin{itemize}
\item To the best of our knowledge this is the first study to utilize group absenteeism as a basis for event detection. By studying different types of group anomalies, either bursts or absenteeism, we demonstrate that these anomalies are indicative of key
events such as civil protests or natural disasters.
\item We incorporate graph wavelets as a mechanism to detect the most anomalous subgraphs at different scales. We demonstrate the power of this approach for social media analytics.
\item We define a graph anomaly index that can be used to determine whether a graph is abnormal. We then apply the graph wavelet to locate the central node and identify the abnormal groups.
%\item We propose a novel two-pass event detection method that uses correlation scores between the group depicting \textit{absenteeism} and the group demonstrating increased activity to probabilistically determine the likelihood of an event.
\end{itemize}

The rest of this paper is organized as follows. Section~\ref{sec:related} reviews related work and existing methodologies and Section~\ref{sec:problem} formalizes the research problem. In Section~\ref{sec:algorithm}, we discuss the graph wavelet formalism for group anomaly detection.
%and subsequently demonstrate how it can be used for two-pass event detection.
Section~\ref{sec:experiment} extensive experiments testing our new approach's effectiveness for real-world event detection, and the paper concludes with a summary of the research in Section~\ref{sec:conclusion}.


\section{RELATED WORK}
\label{sec:related}
\paragraph{Group Anomaly Detection}
Anomaly detection in graphs has been well studied using outlier detection methods~\cite{akoglu2009anomaly}. When considering group concepts, two directions have been explored~\cite{akoglu2015graph}, namely anomalies in unlabeled/plain graphs~\cite{noble2003graph} and those in
attributed graphs. In plain graph anomaly detection, since the only information provided is its structure, features such as distances and communities~\cite{sun2005neighborhood} have been employed to define graph anomalies. In one interesting study~\cite{henderson2010metric}, additional metrics such as vertices, edges, degree, weight, and connected components are incorporated into the detection framework. In attributed graphs, features regarding node behaviors make it possible to create a richer graphical representation, which is usually connected with one or more real-world applications. Other studies, for example~\cite{yu2014glad} define groups based on the term of the role, and model normal groups that follow the same pattern with respect to their role mixture rates. Alternatively, some researchers working on group anomaly detection have built generative models of group anomalies~\cite{xiong2011hierarchical} where the goal is to automatically infer the groups and thus detect group anomalies in a social network. Typically these mixed models suffer from high computational complexity, however, due to the size of the data and are as a result heavily parameterized. In this study, we consider both the graph structure and node features in our proposed graph wavelet based approach for group anomaly detection, as this guarantees that the detected group is automatically compact, with linear computation complexity and scalability.

\paragraph{Event Detection}
Event detection based on LSBNs is a research area that has attracted significant attention in recent years. Traditional approaches focus on capturing the spatiotemporal burstiness of keywords~\cite{lappas2009burstiness,lappas2012spatiotemporal}; Kalman filtering to track the geographical trajectories of hot spots of tweets related to earthquakes~\cite{sakaki2010earthquake}; detecting topics of interest that are coherent within specific geographic regions~\cite{eisenstein2010latent,hong2012discovering,yin2011geographical}; applying clustering-based approaches to search for emerging clusters of documents or terms using predefined similarity metrics that consider factors such as term co-occurrences and social interactions~\cite{aggarwal2012event,sayyadi2009event,watanabe2011jasmine,weng2011event}; and using the notion of compactness of a graph~\cite{rozenshtein2014event} to detect events. Several statistical methods have also been developed, based on Kulldroff's spatial scan statistic~\cite{kulldorff1997spatial}, to detect spatial outliers~\cite{chen2008detecting} and these have been applied to a wide variety of domains including transportation networks, civil unrest forecasting~\cite{zhao2014unsupervised}, and heterogeneous social media graphs~\cite{chen2014non}. The approach we have chosen to address the event detection problem is conceptually different from all the above mentioned studies as it is based on the use of a graph-theoretic framework to detect group anomalies and correlate these to predict future events. Although group absence behavior has been widely studied in the area of organizational behavioral studies~\cite{gaudine2001effects,seamonds1982stress}, it remains largely unexplored in the area of social network analysis. 
%Bearing a close resemblance to group anomaly detection in complex networks, our new detection approach is further distinguished by its focus on groups rather than individuals.

\paragraph{Graph Wavelets}
One of the key challenges facing our research is the need to adapt a detection procedure to encompass both missing and bursty activity groups. To address this issue, we incorporate spectral graph wavelets~\cite{hammond2011wavelets} into our algorithm. This strategy has previously been found to be quite effective for multiscale community mining~\cite{tremblay2014graph}.
Wavelet methods based on spectral graph theory have been applied to wide array of data mining tasks such as community detection, anomaly detection~\cite{calderara2011detecting}, and other machine learning tasks~\cite{shuman_ACHA_2013,ghosh2003wavelet,rustamov2013wavelets,2000wavecluster}. By constructing wavelets over graphs we are able to take advantage of the local information encoded in the graph structure to cluster and identify nodes that are similar to one another in a scale-dependent fashion.

\section{PROBLEM SETTING}
\label{sec:problem}
In this section, we begin by introducing the mathematical notation used in this paper and then proceed to formalize our approach to group anomaly detection. After presenting the problem statement, we provide a brief comparison of our approach with conventional solutions, and review the challenging issues that are relevant to this type of event detection problem.

\subsection{Notation}
\label{sec:notations}
We are given an undirected, weighted graph $\mathbf{G}(V,E;f)$, where $V=\{v_0,v_1,...,v_{N-1}\}$ represents the set of $N$ cities and $E$ refers to the connections between neighboring cities. $W$ is a matrix of non-negative weights associated with each edge, where $e_{ij}\in E$. The function, $f: V \rightarrow {\mathbb{R}}^N$ maps the vertices of graph $\mathbf{G}$, and $f(n)$ stands for the value on the vertex $v_n$. Graph $\mathbf{G}$'s adjacency matrix $\mathbf{A}$ is of size $N\times N$, where each element $a_{ij}$ is represented as:
\begin{equation}
a_{ij} = \left\{ \begin{array}{rl}
 w_{ij} &\mbox{ when $e_{ij}\in {E}$} \\
  0 &\mbox{ otherwise}
       \end{array} \right.
\end{equation}
Here, $\mathbf{A}$ is symmetric since $a_{ij}=a_{ji}$.
Let $d_i=\sum\limits_{v_j \in V}a_{ij}$ be the sum of all edge weights that are incident on $v_i$, and $\mathbf{D}$ be the diagonal matrix denoted as $\mathbf{D}=diag\{d_1,d_2,\ldots,d_N\}$. A Laplacian matrix $\mathcal{L}$ is defined as $\mathcal{L}=\mathbf{D-A}$. It is a symmetric matrix and has real eigenvalues $\lambda_{i}$ such that $0 = \lambda_{0} < \lambda_{1} \leq \lambda_{2} \leq \ldots \leq \lambda_{N-1} = \lambda_{max}$. The complete set of $\mathcal{L}$'s normalized eigenvectors~\cite{bapat2010graphs} $\chi_{i}$ for $i=0,1,2,...,N-1$ is described as:
\begin{equation}
\label{eq:eigenvalues}
\mathcal{L}\chi_{i}=\lambda_{i}\chi_{i}
\end{equation}
The set of eigenvalue and normalized eigenvector pairs is denoted as:
\begin{equation}
\label{eq:spectrum}
\sigma({\mathbf{G}}):=\{(\lambda_l,\chi_l)\}_{l=0}^{N-1}.
\end{equation}$\sigma({\mathbf{G}})$ is also called the graph spectrum of $\mathbf{G}$.




\subsection{Problem Statement}
\label{sec:problemformulation}
We focus on the problem of group anomaly detection from online social networks, based on the absenteeism behavior observed in user activity in geographically proximal communities or group of cities.
Conventionally, this problem can be described as following: \emph{given a graph and \textit{absenteeism score} vector, $\mathbf{G}(V,E;f^t)$ at time interval $t$, select a subset $\Sigma \subseteq V$, such that
\begin{eqnarray}
 \label{eq: problem}
    \Sigma=\underset{P\subseteq V, P \mbox{ is compact}}{\arg\min}\ \ \sum_{v_k\in P} {f(k)}
\end{eqnarray} }
Defining compactness of the selected subset $\Sigma$ is, of course, the key issue here.
A general solution to this problem involves employing a combinatorial optimization method; by defining a constrained objective function over a network one can identify a subset of vertices which minimize the corresponding function~\cite{rozenshtein2014event}. Therefore, Equation~\ref{eq: problem} can be modified as:
\begin{eqnarray}
 \label{eq: problem_conventional}
    \Sigma=\underset{P\subseteq V}{\arg\min}\ \ \sum_{v_k\in P} {f(k)}+\lambda \mu(P),
\end{eqnarray}
where $\mu(P)$ is the compactness penalty function of $P$ (e.g., the sum of distances among
all pairs of the vertices in $P$~\cite{rozenshtein2014event}), and $\lambda$ is the regularization parameter.
However, such methods suffer from the following issues:
\begin{enumerate}
\item Definition of the compactness function $\mu(P)$ is subjective.
%To define and measure the compactness of subset $P\subseteq V$ is challenging, considering the exponential varieties of complex graphs.
\item  Determination of an appropriate regularizer $\lambda$ is difficult, as we do not have sufficient training data for this purpose.
%To determine a suitable regularization parameter $\lambda$ in the objective function is ambiguous, because simply combining multiple physical different concepts in the objective function makes the optima sensitive to $\lambda$.
\item To solve this objective function is often a NP-hard problem~\cite{rozenshtein2014event}, which makes it impractical in many real world applications. Sometimes, even the approximate solutions are of high computation complexity, if there are any.
\end{enumerate}

In contrast, our approach proposes a novel group anomaly algorithm for social networks that is based on spectral graph wavelet theory.
The graph wavelets focus on the intrinsic geometric structure of the graph by transforming each vertex $v_i\in V$, and mining the topological information of both local and global centered vertices to support a multiscale analysis. In addition, the graph wavelet approach identifies anomaly groups that are automatically compact, and provides a fair method at a low computational cost in terms of complexity for identifying abnormal group behavior in broad application scenarios.


\section{ALGORITHMS}
\label{sec:algorithm}

In this section, we first introduce the graph Fourier transform concept, explain the interpretations
of eigenvectors and eigenvalues, and then adapt them to constitute a
newly defined anomaly index for graphs. We then proceed
to describe graph wavelet features such as reconstruction and localization. In Section~\ref{sec:Group_Anomaly_Detection_via_graph_wavelet}, we propose a new group anomaly detection algorithm based on graph wavelets.
%And in section~\ref{sec:Group Absenteeism Event Detection} we present the two-pass event detection algorithm.

\subsection{Graph Fourier Transform}
\label{sec:Graph_Fourier_Transform}
Given a signal $f$ defined on graph $\mathbf{G}$, its graph Fourier transform is considered as the projection of $f$ on the complete set of $\{\chi_l\}_{l=0}^{N-1}$, and is written as~\cite{hammond2011wavelets}:
\begin{equation}
\label{eq:Graph_Fourier_Transform1}
\hat{f}(l)=<\chi_{l},f>=\sum_{i=1}^{N}\chi^*_{l}(i)f(i)
\end{equation}
Since $\{\chi_l\}_{l=0}^{N-1}$ is complete, $f$ can be recovered by its graph Fourier transform coefficients $\hat{f}(l)$ as~\cite{hammond2011wavelets}:
\begin{equation}
\label{eq:Inverser_Graph_Fourier_Transform}
f(n)=\sum_{l=0}^{N-1}\hat{f}(l)\chi_{l}(n)
\end{equation}
Here, $\hat{f}(l)$ is the coefficient of component $\chi_l$.
\subsubsection{eigenvector $\chi_l$}
As an analog with classical signal processing, eigenvector $\chi_l$ is also referred to as
the  frequency of $\mathbf{G}$ by some researchers. In the latter part of this paper, $\chi_l$ will be referred to as the eigenvector or frequency, alternatively. However, unlike the traditional frequency concept in classical signal processing fields, the frequency of $\mathbf{G}$ is a set of discreet vectors with length of $|V|$. Interestingly, like the classical signal Fourier transform, the
Parseval relation~\cite{shuman2015vertex} still holds, i.e.,
\begin{equation}
\label{eq:Parseval}
||\hat{f}||_2^2=||f||_2^2
\end{equation}
Equation~\ref{eq:Parseval} means that the
energy in the vertex domain and frequency domain is equal for any graph signal $f$. Without loss of generality, we assume $||f||_2 =1$.

\subsubsection{eigenvalue $\lambda_l$}
According to the definition of eigenvalue $\lambda_l$  in Equation~\ref{eq:eigenvalues}, the following equation holds:
\begin{equation}
\label{eq:lambda1}
\chi_{l}^T\lambda_{l}\chi_{l}=\chi_{l}^T\mathcal{L}\chi_{l}= \sum_{e_{mn}\in E} w_
{mn}[\chi_{l}(m)-\chi_{l}(n)]^2
\end{equation}Since $\chi_{l}$ is normalized, and $||\chi_{l}||_2 =1$, then,
\begin{equation}
\label{eq:lambda2}
\chi_{l}^T\lambda_{l}\chi_{l}=\lambda_l= \sum_{e_{mn}\in E} w_
{mn}[\chi_{l}(m)-\chi_{l}(n)]^2
\end{equation}
From equation~\ref{eq:lambda2}, we can see that $\lambda_l$ summarizes all the eigenvector deviations on any directly connected vertices $v_m$ and $v_n$ in $\mathbf{G}$. Since each term in the summation of the right-hand side is non-negative, the eigenvectors associated with smaller eigenvalues are smoother; i.e., the component differences between neighboring vertices are
small~\cite{shuman2015vertex}. As the eigenvalue increases, larger differences in neighboring
components of the graph Laplacian eigenvectors are present.
Hence, for larger $\lambda_l$, its corresponding eigenvector, $\chi_l(n)$, has larger deviation among connected vertices. According to the definition of Laplacian matrix $\mathcal{L}$, it is easy to verify that $\lambda_0=0$ since $\mathcal{L}\cdot\vec{\textbf{1}}= 0\cdot\vec{\textbf{1}}$, where $\vec{\textbf{1}}=\{1,1,1,...,1\}$, and $\chi_o(n)=\frac{\vec{\textbf{1}}}{\sqrt{N}}$. Thus, $\chi_o(n)=\frac{\vec{\textbf{1}}}{\sqrt{N}}$ means that $\chi_o(n)$ is constant on each vertex, and that
there is no deviation among any two vertices in $\chi_0(n)$. For this reason, $\chi_0(n)$ is considered as the least abnormal component of $\mathbf{G}$. Similarly, $\chi_{N-1}(n)$ is considered as the most abnormal component of $\mathbf{G}$.

Figure~\ref{fig:graph_G} shows an undirected graph $\mathbf{G_1}$ where each edge's weight is $1$. Figure~\ref{fig:frequency1} shows  $\mathbf{G_1}$'s six eigenvectors distributions along each vertex. We can see that
$\chi_0$ is constant on very vertex, and has the smallest deviations along each edge. $\chi_5$ has the largest deviations, and the difference of $\chi_5$ along each edge is larger than any other eigenvector on average.

\begin{figure}[h]
	\centering
    {
		\includegraphics[height=0.8in] {figures/graph_G.png}
	}
    \vspace{-2mm}
	\caption{Example graph $\mathbf{G_1}$ where all edges' weights are $1$.}
	\label{fig:graph_G}
\end{figure}


\begin{figure}[ht]
	\centering
    \vspace{-3mm}
	\subfigure[]{
		\includegraphics[width= 2.4in,height=0.9in] {figures/frequency.png}
		\label{fig:frequency1}
	}
    \vspace{-3mm}
	\subfigure[]{
		\includegraphics[width= 2.3in,height=0.9in] {figures/g1_gamma.png}
		\label{fig:g1_gamma}
	}
    \vspace{-2mm}
	\subfigure[]{
		\includegraphics[width= 2.45in,height=0.9in] {figures/same_graph.png}
		\label{fig:same_graph}
	}

	\caption{(a): Eigenvector distribution along each vertex in graph $\mathbf{G_1}$.  (b): anomaly index $\gamma_f(l)$ of $f_1=[2,3,4,3,2,1]$ on graph $\mathbf{G_1}$. (c): anomaly index $\gamma_f(l)$ of $f_1=[2,3,4,3,2,1]$  and $f_2=[2,2,-3,4,3,1]$ on graph $\mathbf{G_1}$, where $\gamma_{f_1}=0.905$, and $\gamma_{f_1}=0.073$, labelled in red ovals.}
	\label{fig:f_on_g2}
\end{figure}

\subsection{Global Anomaly Index}
\label{sec:signal_anomaly_on_Graph}

To quantify the anomaly of a vector $f$ defined on a graph $\mathbf{G}$, it's necessary to incorporate the intrinsic structures of $\mathbf{G}$ and $f$. As discussed above, $\hat{f}(l)$ represents the coefficient of frequency $\chi_l$, and $\hat{f}^2(l)$ is considered as the energy of frequency $\chi_l$. In addition, according to equation~\ref{eq:lambda2}, $\lambda_l$ represents the deviation of frequency $\chi_l$ along all the connected vertices. Therefore, in this paper, we define the anomaly index of $\chi_l$ in $f$ as:
\begin{equation}
\label{eq:lambda3}
\gamma_f(l;\mathbf{G})=\lambda_l\hat{f}^2(l)= \lambda_l<f,\chi_l>^2
\end{equation}
$\gamma_f(l;\mathbf{G})$ depends on two parts, frequency $\chi_l$'s deviation sum $\lambda_l$, and its energy $\hat{f}^2(l)$. If the energy $\hat{f}^2(l)$ is small, even if $\lambda_l$ is large, the anomaly index of $\chi_l$ might be small. Obviously, $\gamma_f(0;\mathbf{G})$ is always $0$ since $\lambda_0=0$. Further, we use the maximal value of $\gamma_f(l;\mathbf{G})$ to represent the global anomaly of $f$ on $\mathbf{G}$:
\begin{equation}
\label{eq:lambda4}
\gamma_f(\mathbf{G})=\underset{0 \leq l \leq N-1}{\max}{\gamma_f(l;\mathbf{G})}.
\end{equation}
Here, $\gamma_f(l;\mathbf{G})$ refers to the anomaly extension of $\chi_l$ in $f$ defined on $\mathbf{G}$, instead of implying the anomaly extension of vertex $v_l$.
For brevity, $\gamma_f(l;\mathbf{G})$  and $\gamma_f(\mathbf{G})$ are shortened as $\gamma_f(l)$ and $\gamma_f$, respectively, when $\mathbf{G}$ is known.

Figure~\ref{fig:g1_gamma} plots the anomaly index $\gamma_f(l)$ of $f_1$ on graph $\mathbf{G_1}$, where $f_1=[2,3,4,3,2,1]$. The six markers on the dashed blue are the six eigenvalues of $\mathbf{G}$. The yellow line is $|\hat{f}(l)|$, and the pink line is the anomaly index, $\gamma_f(l)$ for frequency $\chi_l$. Because $\gamma_f(l)$ depends on both $\lambda_l$ and its power $\hat{f}^2(l)$, for the yellow line, even though $\chi_0$ has the strongest power, its deviation $\lambda_0 = 0$, thus $\gamma_f(0)=0$. On the other hand, $\chi_5$ has the largest deviation; but its power $|\hat{f}(5)|^2$ is small, which makes $\gamma_f(5)$ is also small. Considering the $\chi_4$ has a high deviation (eigenvalue) and a strong power of frequency, it has the largest anomaly index. To compare the influence of different $f$ on anomaly index, we show an example in Figure~\ref{fig:same_graph}. Setting $f_1=[2,3,4,3,2,1]$ and $f_2=[2,2,-3,4,3,1]$, we plot their anomaly index $\gamma_{f}$ and energy $|\hat{f}(l)|$ respectively.
The light blue curves stand for anomaly indices and the
yellow curves stand for $|\hat{f}(l)|$. The solid line stands for $f_1$, and the
dashed line stands for $f_2$. As we can see, for high frequency $\chi_l$, $f_1$ has a larger power than $f_2$, and hence a higher anomaly index than $f_2$, where $\gamma_{f_1}=0.905$ and $\gamma_{f_2}=0.073$. This is consistent with that $f_1$ has larger deviations than $f_2$.

As we discussed before, the anomaly index depends on graph structure and $f$. As shown in Figure~\ref{fig:same_graph}, different $f$ might have very different anomaly index because the power of $\chi_l$ distribution is different. Similarly, for the same signal $f$ on two different graphs, it might have very different anomaly indices. Figure~\ref{fig:f_on_g} shows two graphs with the same $f=[1,2,5,2]$. Figure~\ref{fig:new_graph} illustrates the anomaly index of $f$ on $\mathbf{G_2}$ and $\mathbf{G_3}$, where $\gamma_{f}(\mathbf{G_2})=0.073$ and $\gamma_{f}(\mathbf{G_3})=0.235$. (This is because in $\mathbf{G_3}$ because there is no edge connecting $v_2$ and $v_3$, the difference between $f(2)$ and $f(3)$ is not considered as an anomaly.)


{\textbf{Remarks:}}
In this subsection, we have introduced the anomaly index $\gamma_f(l;\mathbf{G})$ to measure the anomaly of $\chi_l$ in $f$ defined on $\mathbf{G}$ by combing the spectrum structure of $\mathbf{G}$ and $f$. $\gamma_f(l;\mathbf{G})$ depends on two parts: (1) the eigenvalue which reflects the deviations of $\chi_l$; (2) the $|\hat{f}(l)|^2$  which represents the power of $\chi_l$ in $f$. $\gamma_f(l;\mathbf{G})$ reflects the anomaly index of $\chi_l$. We use the maximal value of $\gamma_f(l;\mathbf{G})$ to define the anomaly index of $f$, which denotes the global anomaly index of $f$ on $\mathbf{G}$.



\begin{figure}[t]
	\centering
	\subfigure[$\mathbf{G_2}$]{
		\includegraphics[height=0.8in] {figures/f_on_g1.png}
		\label{fig:scale1}
	}
	\subfigure[$\mathbf{G_3}$]{
		\includegraphics[height=0.8in] {figures/f_on_g2.png}
		\label{fig:scale2}
	}
    \vspace{-2mm}
	\caption{$f=[1,2,5,2]$ on two graphs $\mathbf{G_2}$ and $\mathbf{G_3}$.}
	\label{fig:f_on_g}
\end{figure}

\begin{figure}[t]
	\centering
    {
		\includegraphics[width= 3in,height=1in] {figures/new_graph.png}
		\label{fig:distribution2}
	}
    \vspace{-2mm}
	\caption{Anomaly indices of $\mathbf{G_2}$ and $\mathbf{G_3}$.}
	\label{fig:new_graph}
\end{figure}


\subsection{Graph Wavelets}
\label{sec:graph_wavelet}
Classic wavelet formalisms have been
referred to as
mathematical microscopes because of their capability to depict
signal anoamlies at different scales. In the case of complex networks, graph wavelets render the graph with good localization properties both in frequency and vertex (i.e. spatial) domains. Their scaling property allows us to zoom in/out of the underlying structure of the graph.

%It is useful to analyze $f$ by taking into account the intrinsic geometric structure of the graph $\mathbf{G}$. In order to identify and exploit structure of  $f\in \mathbb{R}^N$, the spectral graph $\sigma({\mathcal{L}}):=\{\chi_l\}_{l=0}^{N-1}$ can be used as a dictionary of atoms~\cite{shuman_ACHA_2013}. Thus, $f$ can be decomposed as a linear combination of $\{\chi_l\}_{l=0}^{N-1}$ as
%\begin{equation}
%\label{eq:graph_fourier}
%f(n)= \sum\limits_{l=0}^{N-1}\hat{f}(l)\chi_l(n)
%\end{equation}
%, where
%\begin{equation}
%\label{eq:graph_fourier1}
%\hat{f}(l):= \sum\limits_{n=0}^{N-1}\chi^*_l(n)f(n)
%\end{equation}
%$\chi_l$ is called the Fourier frequency of $f(n)$ based on the graph $\mathbf{G}$, and $\hat{f}(l)$ is the corresponding Fourier coefficient.
%Equation~\ref{eq:graph_fourier1} and Equation~\ref{eq:graph_fourier} are called Fourier transform and inverse Fourier transform, respectively.
%Equation~\ref{eq:graph_fourier1} gives a clear representation of the Fourier components in $f(n)$.

Recall that, from Equation~\ref{eq:Graph_Fourier_Transform1}, the anomaly pattern $\hat{f}(l)$ represents the anomaly components of $f$ from the whole graph perspective. However, information concerning the vertex-location cannot be identified from the Fourier transform. To address this issue, Hammond et al.~\cite{hammond2011wavelets} proposed constructing wavelet transforms functions over the vertices using weighted graphs, described in the following steps:

\begin{enumerate}
\item Define a continuous generating kernel functions $g(x)$ on $\mathbb{R}^+$;
\item Then, select a central vertex $a \in {V}$ and scale $s$, set the frequency coefficients as $g(s\lambda_l)\chi^*_l(a)$ for each frequency component $\chi_l$;
\item Finally, sum up all those frequency components $\chi_l$.
\end{enumerate}
In this way, the graph wavelet at central vertex $a$ is constructed as:
\begin{equation}
\label{eq:graphwaveletdefinition}
\psi_{s,a}(n) = \sum\limits_{l=0}^{N-1}g(s\lambda_l)\chi_l^*(a)\chi_l(n)
\end{equation}
%After setting up the graph wavelet, the wavelet coefficients for $f$ can be defined as
%\begin{equation}
%\label{eq:graph_graphwavelet}
%W_f(s,a)=<\psi_{s,a}, f>=\sum\limits_{l=0}^{N-1}g(s\lambda_l)\hat{f}(a)\chi_l(n)
%\end{equation}
%\paragraph{\textbf{Properties}}
After setting up the graph wavelet, the wavelet coefficients for $f$ can be defined as
\begin{equation}
\label{eq:graph_graphwavelet}
W_f(s,a)=<\psi_{s,a}, f>=\sum\limits_{l=0}^{N-1}g(s\lambda_l)\hat{f}(a)\chi_l(n)
\end{equation}
Similar to classical wavelets, graph wavelets obey following three properties, which are presented in detail in~\cite{hammond2011wavelets}.
 \begin{enumerate}
 \item \textbf{Reconstruction.}
 When the kernel function $g(x)$ satisfies the admissibility condition and $g(0)=0$,  $f(n)$ can be reconstructed by the wavelet coefficients.
\item \textbf{Discretization and Wavelet Frames.} For practical applications, the
scale $s$ of graph wavelet $\psi_{s,a}$ should be sampled with a finite number of scales. Given a real valued function $h(x)$ satisfying
\begin{equation}
\hat{h}(\omega) = \sqrt{\int_\omega^\infty\frac{|\hat{g}(\omega')|^2}{\omega'}d{\omega'} },
\end{equation}
where $\hat{g}$ and $\hat{h}$ are the classical Fourier transform of $g(x)$ and $h(x)$, the scaling function $\phi_{a}(n)$ can be generated as:
\begin{equation}
\label{eq:graphscaledefinition}
\phi_{a}(n) = \sum\limits_{l=0}^{N-1}h(\lambda_l)\chi_l^*(a)\chi_l(n)
\end{equation}
Accordingly, the scaling coefficients are defined as
\begin{equation}
S_f(a)=<\phi_a,f>
\end{equation}
Using scale set $\Theta:=\{s_j\}_{j=1}^J$, the discretized graph wavelet set $\{\psi_{s_j,a}\}_{j=1}^{J}$ $_{a=0}^{N-1}$, and scaling function set $\{\phi_a\}_{a=0}^{N-1}$ constitute a frame~\cite{hammond2011wavelets}.
According to frame theory~\cite{daubechies1992ten}, $f\in \mathbb{R}^N$ can be reconstructed from those $NJ+J$ wavelet and scaling coefficients as
\begin{equation}
\label{eq:frame}
f(n)=\sum_{a=v_0}^{v_{N-1}}[\sum_{j=1}^{J}W_{f}(s_j,a)\psi_{s,a}(n)+S_f(a)\phi_{a}(n)].
\end{equation}For brevity, we assume that
\begin{equation}
\phi_a(n)=\psi_{s_0,a}(n),
\end{equation} and
\begin{equation}
S_f(a)=W_f(s_0,a).
\end{equation}Therefore, equation~\ref{eq:frame} can be written as
\begin{equation}
\label{eq:frame2}
f(n)=\sum_{a=v_0}^{v_{N-1}}\sum_{j=0}^{J}W_{f}(s_j,a)\psi_{s,a}(n).
\end{equation}In the later part of this paper, we do not differentiate between
scaling coefficient and wavelet coefficient.
A detailed algorithm and treatment concerning the choice of $\Theta$ can be found in~\cite{hammond2011wavelets}.


\begin{figure*}[t]
	\centering
	\subfigure[wavelet $\psi_{s_1,a}$]{
		\includegraphics[width=1.4in, height=1.1in] {figures/wavelet1-2.png}
		\label{fig:scale1}
	}
	\subfigure[wavelet $\psi_{s_2,a}$]{
		\includegraphics[width=1.4in, height=1.1in] {figures/wavelet2.png}
		\label{fig:scale2}
	}
\subfigure[$f(n)$ vs vertices]{
		\includegraphics[width=1.4in, height=1.1in] {figures/wavelet3.png}
		\label{fig:scale3}
	}
\subfigure[$W_f(s,a)$ vs scale $s$]{
		\includegraphics[width=1.5in, height=1.1in] {figures/wavelet4.png}
		\label{fig:scale4}
	}
    \vspace{-2mm}
	\caption{Graph wavelet scale and graph wavelet coefficient.}
	\label{fig:graphwaveletscale}
\end{figure*}




\item \textbf{Localization in vertex domains.} Given a central vertex $v_a$ and its graph wavelet $\psi_{s,a}(n)$, suppose the kernel function $g$ is $K+1$ times continuously differentiable, let $v_n$ be an vertex of $\mathbf{G}$ with $d_G(n,a)>K$, then there exists constants $D$ and $\beta$, such that
\begin{equation}
\label{equ:waveletbound}
\frac{|\psi_{s,a}(n)|}{||\psi_{s,a}||}\leq D \beta
\end{equation} for all $s<\beta$.
$d_G(n,a)$ is the shortest path distance, which is the minimum number of edges in any path that connect vertices $v_n$ and $v_a$~\cite{hammond2011wavelets}. Equation~\ref{equ:waveletbound} shows for any vertex $v_n$ that is far away from center vertex $v_a$ ($d_G(n,a)>K$), $\frac{|\psi_{s,a}(n)|}{||\psi_{s,a}||}$ is upper bounded by $D\beta$. In other words, for vertex $v_n$ which is far away form vertex $v_a$, its wavelet value is linearly attenuated by scale $s$.  When the scale $s$ is small, their wavelet value of marginal vertices will be vanished quickly. The marginal vertices are those which satisfy equation~\ref{equ:waveletbound}. All the other vertices are called kernel vertices, denoted by $\mathcal{K}(s,a)$. Obviously, $\forall v_n \in \mathcal{K}(s,a)$,  $d_G(n,a)\le K$. Thus $\mathcal{K}(s,a)$ is automatically compact.
Figure~\ref{fig:graphwaveletscale} shows two graph wavelets centered on the same vertex $a$, but with two different scales, $\psi_{s_1,a}$ and $\psi_{s_2, a}$, where $s_1<s_2$. The length of the vertical bar on each vertex denotes its graph wavelet value. The highlighted areas denote the kernel vertices ($d_G(n,a)\le 1$), and the others are marginal vertices. We can see that the wavelet values  on marginal vertices in Figure~\ref{fig:scale1} are smaller than those in Figure~\ref{fig:scale2}. Figure~\ref{fig:scale3} is $f$'s distribution along each vertex, and Figure~\ref{fig:scale4} shows the wavelet coefficients with center node $a$ for different scales, which indicates that
$W_f(s_2,a)$ has the largest value, and $W_f(s_3,a)$ with the smallest.
 \end{enumerate}



\subsection{Group Anomaly Detection via Graph Wavelets}
\label{sec:Group_Anomaly_Detection_via_graph_wavelet}
According to Equation~\ref{equ:waveletbound}, when $s$ is small, the weights of the marginal vertices are severely attenuated.
Essentially, $W_f(s,a)$ is equivalent to the sum of $f$ with large weights on kernel vertices, and small weights on marginal vertices.
%, and can also be treated as a similarity between $f$ and $\psi'_{s,a}$.
When $f$ is of uniformly large negative/positive values on kernel vertices, then $W_f(s,a)$ will be a large negative/positive value with scale $s$.

The localization property of graph wavelets makes them appropriate for group anomaly detection since they automatically identify the kernel vertices from marginal vertices. These kernel vertices form a compact subset since each one of them is close to the same center vertex $a$, which avoids the compactness constraint condition in Equation~\ref{eq: problem_conventional}, thus reducing
its computational complexity greatly. We propose our group anomaly detection algorithm based on graph wavelets in Algorithm~\ref{algo:event_detection1}. It iterates $NJ+J$ times, and each iteration selects a vertex as the center node, and computes the wavelet coefficient $W_f(s_j, a)$ with $J+1$ scales. When $W_f(s_j, a)$ is larger than some pre-set threshold $\omega_{th}$, it considers the corresponding kernel vertices, $\mathcal{K}(a)$, as an abnormal burst group. Similarly, when $W_f(s_j, a)$ is smaller than $-\omega_{th}$, it considers $\mathcal{K}(a)$ as an abnormal absenteeism group. The computational complexity is $O(J|V|^2)$.

\begin{algorithm}[ht]
\centering
\captionsetup{font=scriptsize}
\caption{Group Anomaly Detection using Graph Wavelets}
{\footnotesize \begin{algorithmic}[1]
\STATE {\bf Input:} graph and absenteeism score vector $\mathbf{G}(V,E;f^l)$ at time interval $l$, wavelet threshold $\omega_{th}$.
\STATE {\bf Output:} abnormal burst group set $\mathcal{I}^{bur}$ and absenteeism group set $\mathcal{I}^{abs}$.	\STATE{compute graph spectrum $\sigma{(\mathbf{G})}$};
\STATE{set graph wavelets $\psi_{s,a}(n)$ and scales set $\{s_j\}_{j=0}^J$ for all $a\in V$};
\FORALL {center node $a\in V$ and $s_j \in \{s_j\}_{j=0}^J$}
	    \STATE{compute $W_f(s_j, a)$};
		\IF {$W_f(s_j, a) \ge \omega_{th}$}
		    \STATE{add group $\mathcal{K}(s_j,a)$ to $\mathcal{I}^{bur}$}
	    \ENDIF
	
		\IF {$W_f(s_j, a)\le -1*\omega_{th}$}
		    \STATE{add group $\mathcal{K}(s_j,a)$ to $\mathcal{I}^{abs}$}
	    \ENDIF	
	
\ENDFOR	
\RETURN {abnormal burst group $\mathcal{I}^{bur}$ and absenteeism group set $\mathcal{I}^{abs}$.}
\end{algorithmic}}
\label{algo:event_detection1}
\end{algorithm}


{\textbf{Remarks:}}
\begin{enumerate}
\item Graph wavelets form a frame where the function $f$ can be reconstructed by their coefficients.
As long as the scale level $J$ is high enough, $f$ can be well decomposed into the frame basis. Thus, using graph wavelets to exploit the structure of functions defined on graphs is much more reasonable.
\item Graph wavelets transform selected kernel vertices, $\mathcal{K}(s,a)$, that are close to the central vertex $a$, and attenuate the impact of other marginal vertices that are far away from $a$. The abnormal group selected by graph wavelet approach is automatically compact, and circumvent high computational complexity, which makes is easily adaptable to a wide variety of application scenarios.
\item Graph wavelets are able to identify abnormal burst groups and absenteeism groups simultaneously without extra computation cost.
\end{enumerate}




\section{EXPERIMENTAL RESULTS}
\label{sec:experiment}
This section discusses the application of our approach for detecting group anomalies. We begin by briefly describing the dataset used for our experiments in Section~\ref{sec:data_collection} and then move on to discussing the implementation details of how the graph $\mathbf{G}$ is assembled; we construct the graph wavelets $\psi_{s,a}$ in Section~\ref{sec:experimental_setup}. The following section presents the group anomaly detection performance for identifying protest events. In Section~\ref{sec:highlighted_results}, we describe three case studies that illustrate how the graph wavelet model is able to capture absenteeism events such as disaster scenarios.


\subsection{Data Collection and Preprocessing}
\label{sec:data_collection}
The study described in this paper uses tweets geolocated to Latin America and
collected over a period of two years
(Jan 2013 to Dec 2014).
We query Datasift's streaming API to collect tweets that also have meta-information including geotag bounding boxes (structured geographical coordinates), Twitter places (structured data), user profile location (unstructured, unverified strings), and `mentions information' about locations present in the body of the tweet.
Typically, we found that
the number of tweets with readily available geo-coordinates is too low for conducting meaningful experiments.
To circumvent this drawback, we use the geo-enrichment algorithm described in~\cite{ramakrishnan2014beating}.
This algorithm uses a gazetteer-based approach to look-up location names and geo-coordinates.
To identify location-specific tweets, we configure the geocoding tool to first consider the tweet's text for mentions of place names and geographical landmarks (e.g., say, Plaza de la Independencia (Quito, Ecuador)).
In cases when no geographical location was found in the tweet text, it then proceeds to process the geographical coordinates and the self-reported location string in user's profile metadata.
Using this geocoding tool, we were able to extract tweets corresponding to $598,300$ unique locations from Latin America.

%To prepare a dataset of ground truth events for our study, we focused on specific types of disruptive societal events, such as natural disasters.
%We assume that such events are the predominant reasons that can cause group absenteeism on social networks.
%To discern when major events occurred, we retrieved records of natural disaster related events involving earthquakes, floods, and landslides from European Emergency Response Coordination Center(ERCC)~\footnote{http://erccportal.jrc.ec.europa.eu/} and World Top Stories Timeline~\footnote{http://www.mapreport.com/}.

\begin{figure}[t]
	\centering
	\subfigure[]{
		\includegraphics[width=1.4in,height=1.3in] {figures/brazil_graph_nn_5.png}
		\label{fig:brazil_graph_nn_5}
	}
	\subfigure[]{
		\includegraphics[width=1.4in,height=1.3in] {figures/brazil_zscore_31.png}
		\label{fig:brazil_zscore_31}
	}
    \vspace{-2mm}
	\caption{(a) Brazil's 5-nearest-neighbor graph: 5321 cities, where
all edges' weights are $1$. (b) Brazil's z-score distribution on July 31, 2013. The color bar shows the scale of z-score.}
	\label{fig:knn_zscore}
\end{figure}

\subsection{Experimental Setup}
\label{sec:experimental_setup}

\paragraph{Graph Setup}
Each city $v_i$'s location is represented by its geographical coordinate pair $lat_i$ and $lon_i$. Instead of using the real physical distance, we define the distance of any two cities $v_i$ and $v_j$ as $d_{ij}=\sqrt{(lat_i-lat_j)^2+(lon_i-lon_j)^2}$. We setup graph $G$ as a $k$ neighbors graph, which means  each city is only  connected to its $k$-nearest-neighbors. In this paper, we set $k=5$, and all the edges' weights  in $G$ are 1. Figure~\ref{fig:brazil_graph_nn_5} shows Brazil's $5$ nearest-neighbor graph with 5321 cities.


\begin{figure}[t]
	\centering
	\subfigure[wavelet $\psi_{s_1,a}$]{
		\includegraphics[width=1.4in] {figures/s1.png}
		\label{fig:Brazil_W_coeff_date31_s3}
	}
	\subfigure[wavelet $\psi_{s_2,a}$]{
		\includegraphics[width=1.4in] {figures/s2.png}
		\label{fig:Brazil_W_coeff_date31_s5}
	}
    \vspace{-2mm}
	\caption{Graph wavelets with center city $v_{83}$. $s_1$ = 1.31, $s_2$ = 0.68.}
	\label{fig:graphwaveletscale}
\end{figure}


%\begin{figure}[t]
%	\centering
%	\subfigure[$W_f(s_1,a)$]{
%		\includegraphics[width=1.2in,height=1.1in] {figures/wavelet_coeff_s1.png}
%		\label{fig:Brazil_W_coeff_date31_s3}
%	}
%	\subfigure[$W_f(s_2,a)$]{
%		\includegraphics[width=1.2in,height=1.1in] {figures/wavelet_coeff_s6.png}
%		\label{fig:Brazil_W_coeff_date31_s5}
%	}
%	\caption{Graph wavelet coefficient $W_f(s_1,a)$ and $W_f(s_2,a)$.}
%	\label{fig:graphwaveletcoefficient}
%\end{figure}




\paragraph{Absenteeism Score}
Considering that
the tweet volume $X$ varies vastly among cities, instead of using $X$ itself, we use the normalized value of z-score as absenteeism score, which is defined as:
\begin{equation}
\label{eq:Z_score}
\textrm{z-score} = \frac{X-\mu}{\sigma}
\end{equation}where $\mu$ is the mean value of the previous $30$ day tweets volume and $\sigma$ is the corresponding standard deviation. As shown in Figure~\ref{fig:brazil_zscore_31}, different
node colors denote different z-score values.

\paragraph{Kernel function $g(x)$ and scaling function $h(x)$}
Our choice for the wavelet generating kernel function, $g(x)$, and scaling function $h(x)$ is motivated by our goal to achieve scale-dependent localization. We follow the kernel function setting in~\cite{hammond2011wavelets}, which behaves as a monic power near the origin, and has power law decay for large x. $g(x)$ and $h(x)$ are set as:
\begin{equation}
g(x) = \left\{ \begin{array}{rl}
 x &\mbox{ for $x<1$} \\
s(x) &\mbox{ for $1\leq x \leq 2$} \\
 2x^{-1} &\mbox{ for $x>2$} \\
       \end{array} \right.
\end{equation} where $s(x)=-5+11x-6x^2+x^3$.
\begin{equation}
h_{x}= 1.385\, exp(-(\frac{20x}{0.6\lambda_{max}})^4)
\end{equation}
The scale set $\{s_j\}_{j=1}^J$ is selected to be equally logarithmically spaced between the minimum and maximum scales $s_1$ and $s_J$, which are defined in~\cite{hammond2011wavelets}. We set $J=6$ in the experiment. Figure~\ref{fig:graphwaveletscale} shows two different scaled wavelets on Brazil's $5$-nearest-neighbor graph. Comparing Figure~\ref{fig:Brazil_W_coeff_date31_s3} with Figure~\ref{fig:Brazil_W_coeff_date31_s5}, we can see that, when scale increases, more cities (with deeper color) are selected. 
%Figure~\ref{fig:graphwaveletcoefficient} shows the corresponding wavelet coefficients.
We also try another kernel function, i.e. the Mexican hat function, and find that as long as the kernel function monotonicity is the same,
the differences in wavelet coefficients are negligible.

\paragraph{Anomaly index $\gamma_f(G)$ and $\omega_{th}$}
We claim that the event frequency $\eta$ is linear to $\gamma_f(\mathbf{G})$, described as
\begin{equation}
\label{eq:linear_equation}
\eta = k_0*\gamma_f(\mathbf{G}) + k_1
\end{equation}We use historical data to train $k_0$ and $k_1$ by least square error criterion. Once we know
$k_0$ and $k_1$, given a new $\gamma_f'(\mathbf{G})$, the event number is estimated as $m=\left \lceil \eta' \right \rceil$. Subsequently the
threshold $\omega_{th}$ is set as the $m_{th}$ largest $W_f(s_j,a)$, for all $a\in V$, $0\le j \le J$.


\begin{table}[bt] %!htp
\renewcommand{\arraystretch}{1.1}
\caption{\label{table:models_compare} The performance of graph wavelet vs. baseline and Z-score.}
\scriptsize
\centering
\begin{tabular}{ l | l |l | l | l}
\hline
\textbf{Country} & \textbf{Method}& \textbf{Precision}  & \textbf{Recall}  & \textbf{F-measure} \\
\hline
Brazil & Baseline & 0.052 &0.104 & 0.060\\
       & Z-score & 0.117&0.307 & 0.159 \\
 & Graph wavelet& 0.404 &0.262 & 0.292 \\
\hline
Mexico & Baseline & 0.074 &0.124 & 0.090 \\
       & Z-score & 0.221 &0.147 & 0.168 \\
 & Graph wavelet& 0.397 &0.384 & 0.408 \\
\hline
Venezuela & Baseline & 0.078 &0.053 & 0.059 \\
       & Z-score & 0.197 &0.197 & 0.189 \\
 & Graph wavelet& 0.292 &0.554 & 0.355 \\
\hline
\end{tabular}
\end{table}


\begin{figure}[t]
	\centering
	\subfigure[Brazil]{
		\includegraphics[width= 2.6in,height=1.4in] {figures/performance_compare_bar_graph_brazil.png}
		\label{Brazil_performance}
	}
    \vspace{-2mm}
	\hfill
	\subfigure[Mexico]{
		\includegraphics[width= 2.6in,height=1.4in] {figures/performance_compare_bar_graph_mexico.png}
		\label{Mexico_performance}
	}
    \vspace{-2mm}
	\hfill
	\subfigure[Venezuela]{
		\includegraphics[width= 2.6in,height=1.4in] {figures/performance_compare_bar_graph_venezuela.png}
		\label{Venezuela_performance}
	}
    \vspace{-2mm}
	\caption{Brazil, Mexico and Venezuela protest detection performance.}
\label{fig:threecountry_performance}
\end{figure}



\subsection{Performance}
The data for this experiment was gathered for three countries experiencing major protest events, namely Brazil, Mexico and Venezuela, from Jan 2013 to Dec 2014. Taking the Gold Standard Report (GSR)~\cite{ramakrishnan2014beating} as representing ground truth, we applied our new graph wavelet approach as follows. For each day, we determine whether there is any anomaly. If there is, we
identify the group of anomalous cities and compare this set
with the GSR to determine if the selected cities actually experienced protest events on that day and thus show how many of the model's predictions matched the ground truth and how many did not. We use recall, precision, and the F-measure to evaluate the model's performance. To evaluate the effectiveness of our new graph wavelet approach, we also compared the results with those obtained using intuitive approaches such as frequency based random assignment, referred to here as the baseline model, and Z-score based selection methods. The baseline model was built according to the historical protest records for each city and thus the model's predictions of the future occurrence of protests were based on frequency. The z-score approach entails
selecting the group of cities whose z-score crosses some threshold, say $|Z-score|>3$.


We compared the performance of these three models over the two year test period; the overall results are shown in Table~\ref{table:models_compare}. Generally speaking, the new graph wavelet approach exhibited better precision, recall, and F-measure scores than the baseline model across all three countries. The mean F-measure for the graph wavelet detection across models and countries is greater than that achieved by either of the other prediction models. Interestingly, the graph wavelet approach appears to operate at different efficiency levels for each country. From Figure~\ref{fig:threecountry_performance} we can see that the graph wavelet model has a much higher recall in Venezuela than in Brazil, and an inferior quality of event detection in Mexico compared to Venezuela.

\subsection{Case Studies}
\label{sec:highlighted_results}

\begin{figure}[t]
	\centering
	\subfigure[]{
		\includegraphics[width=0.7in,height=0.8in] {figures/Chile_absent_zscore_3.png}
		\label{fig:absent_Chile_score}
	}
	\hfill
	\subfigure[]{
		\includegraphics[width=0.7in,height=0.8in] {figures/Chile_absent_wavelet_3.png}
		\label{fig:absent_Chile_wavelet}
	}
	\hfill
	\subfigure[]{
		\includegraphics[width=0.7in,height=0.8in] {figures/Chile_burst_zscore_3.png}
		\label{fig:burst_Chile_score}
	}
	\hfill
	\subfigure[]{
		\includegraphics[width=0.7in,height=0.8in] {figures/Chile_burst_wavelet_3.png}
		\label{fig:burst_Chile_wavelet}
	}
	\subfigure[]{
		\includegraphics[width=1.4in,height=0.7in] {figures/earthquake_example_10min_circle.png}
		\label{fig:earthquake}
	}
	\subfigure[]{
		\includegraphics[width=1.4in,height=0.7in] {figures/earthquake_cloud.png}
		\label{fig:earthquake-cloud}
	}
	\caption{Iquique Earthquake, Chile. (a-d) plots show differences in distributions of absenteeism score and wavelet coefficients calculated at 8:45 PM, April 1, 2014 (a-b) involving group absenteeism and later when burst in activity is captured at 11:00 AM, April 2, 2014 (c-d), respectively; (e) Tweet time series for Iquique on April 1, 2014; (f) Word cloud of tweets which mention `Iquique'.}
\label{fig:case1_wavelet}
\end{figure}



\textbf{Case study 1: Iquique Earthquake, Chile.}
On April 1, 2014 at around 8:46 PM (local time) a large earthquake struck off the coast of Chile, northwest of the port city of Iquique. We show the distribution of absenteeism scores and normalized wavelet coefficient values of the graph wavelets from the beginning of this event and throughout the subsequent 24 hour period. As shown in Figure~\ref{fig:absent_Chile_score}, we can clearly see absenteeism behavior, where the scores are dominated by very low (blue spectrum) Z-score values (indicating high absenteeism). Likewise, Figure~\ref{fig:absent_Chile_wavelet} depicts low coefficient values for the northern regions of Chile, where the impact of the earthquake was most significant. As the news of earthquake spread throughout the next day, user activity on social media increased. This bursty behavior is clearly visible on April 2nd, at around 11:00 AM. Figure~\ref{fig:burst_Chile_score} shows that the Z-scores increase (red spectrum) significantly and the coefficient value distribution (Figure~\ref{fig:burst_Chile_wavelet}) of the graph wavelets for northern regions of Chile are also in the red spectrum. The graph wavelet distributions in Figures~\ref{fig:absent_Chile_wavelet}~\ref{fig:burst_Chile_wavelet} show that the kernel area of the absenteeism/burst wavelets cover most large negative/positive values. In this way, the wavelets identify the abnormal negative/positive groups in absent/burst time intervals, respectively. Furthermore, a high correlation score of 0.726 was calculated for the wavelets from absenteeism and bursty periods of this episode, indicating a strong connection between the burst in activity and the previously observed absenteeism, signaling an event was detected.



\begin{figure}[t]
	\centering
	\subfigure[]{
		\includegraphics[width=0.7in, height=0.85in] {figures/Venuze_absent_zscore_3.png}
		\label{fig:absent_Venezuela_score}
	}
	\subfigure[]{
		\includegraphics[width=0.7in,height=0.85in] {figures/Venuze_absent_wavelet_3.png}
		\label{fig:absent_Venezuela_wavelet}
	}
	\subfigure[]{
		\includegraphics[width=0.7in,height=0.85in] {figures/Venuze_burst_zscore_3.png}
		\label{fig:burst_Venezuela_score}
	}
	\subfigure[]{
		\includegraphics[width=0.7in,height=0.85in] {figures/Venuze_burst_wavelet_3.png}
		\label{fig:burst_Venezuela_wavelet}
	}
	\subfigure[]{
		\includegraphics[width=1.4in,height=0.7in] {figures/Veneuela_power_count_all.png}
		\label{fig:power}
	}
	\subfigure[]{
		\includegraphics[width=1.4in,height=0.7in] {figures/power-English.png}
		\label{fig:power-cloud}
	}
	\caption{Power Outage in Venezuela. (a-d) plots show differences in distributions of absenteeism score and wavelet coefficients calculated at 7:40 PM, December 2, 2013 (a-b) involving group absenteeism and later when burst in activity is captured at 8:45 PM in the same day (c-d), respectively; (e) Time series of tweets volume on December 2, 2013; (f) Word cloud of tweets mentioning `Caracas'.}
\label{fig:case2_wavelet}
\end{figure}


The graph wavelets generated during the absenteeism time period Figure~\ref{fig:absent_Chile_wavelet} have a central node located in the city of `Iquique'. Looking at the time series (Figure~\ref{fig:earthquake}) of Twitter activity for Iquique and the associated word clouds (see Figure~\ref{fig:earthquake-cloud}), we can see how events unfolded during the course of the earthquake. Strong absenteeism is observed from 8:45 PM to 9:20 PM. Examining user mobility via their geotagged tweets from the city of Iquique, on April 1, 2014, the user mobility fraction had increased by 15.4\%.


\textbf{Case Study 2: Massive power outage in Venezuela.}
A massive power outage in Venezuela plunged several major cities, including the capital city Caracas, into darkness around 7:40 PM (local time) on December 2, 2013.
News media reported\footnote{http://www.usatoday.com/story/news/world/2013/12/\hskip0ex 02/\hskip0ex power-failure-caracas-venezuela/3823327/}, that the power outage lasted for 10-15 minutes, and the people of Caracas quickly took to the streets to protest.
This action at the beginning of the episode coincides with the absenteeism period detected by our algorithm.
The scatter plots showing the distribution of absenteeism scores and wavelet coefficients (Figures~\ref{fig:absent_Venezuela_score},~\ref{fig:absent_Venezuela_wavelet}) indicate that most of the low values are less than $0$.
Shortly after the absenteeism, we detected a huge burst in activity around 8:45 PM, signaled by the increased z-scores (low absenteeism) and coefficient values (Figures~\ref{fig:burst_Venezuela_score},~\ref{fig:burst_Venezuela_wavelet}). A correlation score of 0.617 was calculated when comparing the graph wavelets from the absentee and burst periods.


The absenteeism related graph wavelets indicate that the city of Caracas was the central node. Taking a close look at the Twitter volume and Tweets from Caracas and surrounding cities, there is a sharp decline in user activity around 7:40 PM and then a huge spike starting at 8:45 PM. The word clouds for the Tweet content show a very similar story, with dominant words being `light' and `blackout'; the Spanish phrase `sin luz', which means `no light', became a trending hashtag \#sinluz on Twitter.



\begin{figure}[t]
	\centering
	\subfigure[]{
		\includegraphics[height=0.9in] {figures/Argentina_absent_zscore_3.png}
		\label{fig:absent_Argentina_score}
	}
	\subfigure[]{
		\includegraphics[height=0.9in] {figures/Argentina_absent_wavelet_3.png}
		\label{fig:absent_Argentina_wavelet}
	}
	\subfigure[]{
		\includegraphics[width=1.3in, height=0.7in] {figures/holiday-cloud.png}
		\label{fig:holiday-cloud}
	}
	\caption{The Christmas Day in Argentina: (a-b) plots show distributions of (a) absenteeism score and (b) wavelet coefficients calculated on December 25, 2013; (c) Time series comparing absenteeism score and user mobility corresponding to tweets between December 5 - 25, 2013.}
\label{fig:case3_wavelet}
\end{figure}


\textbf{Case Study 3: Christmas Day.}
As noted earlier, absenteeism behavior may not always lead to a spike in activity. For example, our model detected strong absenteeism in social media activity for major holidays such as Christmas Day that was not followed by a bursty period in Twitter activity. This is likely because people tend to travel to visit family during the holidays. This is supported by low values of z-scores or high absenteeism in Figure~\ref{fig:absent_Argentina_score} and wavelet coefficients in Figure~\ref{fig:absent_Argentina_wavelet} with respect to Argentinian Tweets on December 25, 2013. Hence, no subsequent burst period was detected for this event. Interestingly as Christmas Day approached, Figure~\ref{fig:holiday-cloud} shows that user mobility gradually increases and the z-score decreases, signaling greater absenteeism. We used Pearson's correlation coefficient to measure the two time series and found a correlation score of -0.94.

\section{Discussion}
\label{sec:conclusion}
Previous research has demonstrated the importance of burst detection in Twitter. In this study, we argue that group absenteeism can also be vital for detecting disruptive societal events. Modeling absenteeism is crucial because it can serve as a surrogate signal for event detection. For example, in the case of the Iquique earthquake, our new algorithm detected absenteeism behavior on Twitter that was closely followed by a spike in user activity. Unlike traditional event detection methods, which identify real time events only after they have occurred because the burst signal must first be identified, an absenteeism signal can be observed much earlier, thus providing greater foresight into future events. This means that our proposed approach offers a significant advantage over current strategies that focus solely on modeling spike or burst related patterns for event detection.

%Disruptive events which cause Twitter absenteeism, but also render burst detection methods less useful. In the case of the Natal protest event, a large portion of people were walking on the street to protest, and the city's tweet absenteeism score reached a minimum. During the Brazil floods, the tweets tended to become inactive as the severity of the floods increased. It reached the lowest point when the flood was at its worst. In these two cases in particular, using a burst signal alone it can be difficult to identify such events.

Existing approaches for event detection also
suffer from an inherent latency in their detection process. This is because they are based on the use of bursty signals from abnormal activity on social networks, but miss the absenteeism signal that often precedes these bursts. Our approach addresses this shortcoming by successfully modeling the
`lull before the storm'. In this study we defined an absenteeism score for groups of cities within the Twitter network and apply it to construct wavelet transforms that not only
detect anomalous subgraphs (including both burst and absenteeism groups) at different scales, but can also be used to identify the geographical focal point of the anomaly. This localization property of graph wavelets guarantees that the selected groups are compact automatically. The identified abnormal groups have been verified using real-world datasets and proven to be indicative of
events such as civil protests or natural disasters.

%In this work, we have presented a systematic and unified framework for detecting, identifying event's location and distinguishing anomalous groups in Twitter.

%From the three case studies, we have shown that the initial phase in the evolution of a disruptive, event is characterized by group absenteeism behavior. This behavior is further underlined by an increase in user mobility. As in the case of ``Christmas Day" event we observed absenteeism from Argentina Twitter users in days leading to December 25th was characterized by increased mobility (inferred from geolocated tweets).

%In future work, we plan to extend our detection model to capture extent of an event's influence over network. Another interesting extension of our work would be to include absenteeism as feature to classify event of different nature (disruptive vs non-disruptive).

%Graph wavelet based approach, considering both the graph structure and the vector f;
%Define an anomaly index of f�s distribution on G;
%Identify abnormal locations using graph wavelet;
%Detect absent and burst groups simultaneously.

%\section*{Acknowledgment}
%
%
%The authors would like to thank...


%%%%%%%%%%%%%%%%%%%%%%%%%%%% end %%%%%%%%%%%%%%%%%%%%%%%%%
%\begin{table*}[th] %!htp
% \renewcommand{\arraystretch}{1}
% \caption{\label{table:list_events} Selected major events in South America countries}
% \scriptsize
% \centering
% \begin{tabular}{ p{0.5cm}| p{2cm} | p{2.2cm} | p{2.2cm} | p{2.2cm} | p{2.5cm} | p{3cm} }
%  \hline
%  \textbf{No.} & \textbf{Events}& \textbf{ Absenteeism } & \textbf{Response time} & \textbf{Correlation}&\textbf{Central location} \\ [1ex]
%  \hline
%        1& Earthquake & 8:45 PM & 3 hours & 0.73 &Iquique, Chile\\
%        2& Blackout & 7:40 PM & 1 hour & 0.81& Caracas, Venezuela \\
%        3& holiday & one day & 2 days & 0.33 & \\        \hline
% \end{tabular}
%\end{table*}


%Public holidays are typical events causing group absenteeism. One of the most dominant reason is, during public holidays, especially long-time holiday, people tend to travel, which resulting in a high level of local user mobility, and users' mobility will cause Twitter absenteeism accordingly.
%We calculating more cases Pearson's correlation, and plot their distribution in Figure~\ref{fig:pearson}, of which the median value of correlation score is -0.88, and the average correlation score is -0.79. We can see the user mobility plays a forceful role in influencing Twitter absenteeism.

%\subsection{Performance}
%We use the data set on February 27, 2014, and set the time window as one day. We plot the comparison results from two aspects:  running time complexity, and parameter sensibility in figure~\ref{fig:performance},~\ref{fig:running_time},~\ref{fig:sensibility}.
%\begin{figure}[ht]
%	\centering
%	\subfigure[matrix]{
%		\includegraphics[width=1.55in,height=1in] {figures/performance1.png}
%		\label{fig:performance1}
%	}
%	\subfigure[graph]{
%		\includegraphics[width=1.55in,height=1in] {figures/performance2.png}
%		\label{fig:performance2}
%	}
%	\caption{running time vs input parameter.}
%	\label{fig:performance}
%\end{figure}
%\paragraph{Running time}From Figure~\ref{fig:performance}, we can see that the running time of minimal matrix approach increases extremely fast when $A$ is larger than 0.09. While in graph wavelet approach, the increasing speed is much stable as $d_{th}$ increases. This is because minimal matrix approach's time complexity is in proportion to $A^2$, while graph wavelet approach's time complexity is proportional to $d_{th}$. From Figure~\ref{fig:running_time}, we can see clearly that for the minimal matrix algorithm, the running time complexity also increase sharply with the input size $n$, while in graph wavelet approach, the increase speed is moderate. This is because the minimal matrix approach's timing complexity is $O(N^3)$, while graph wavelet approach's time complexity is $O(N^2)$. Thus, the graph wavelet approach is better  than minimal matrix approach in term of running time for a larger absenteeism group.
%\begin{figure}[h]
%	\centering
%	\subfigure[matrix]{
%		\includegraphics[width=1.55in,height=1in] {figures/running_time1.png}
%		\label{fig:running1}
%	}
%	\subfigure[graph]{
%		\includegraphics[width=1.55in,height=1in] {figures/running_time2.png}
%		\label{fig:running2}
%	}
%	\caption{Running time vs input size.}
%	\label{fig:running_time}
%\end{figure}
%\paragraph{Parameter sensibility}In minimal matrix approach, set the input parameter as $A$, and the optimal absenteeism group as $P_{min}$. When $A$ is changed to $A$', the optimal absenteeism group is changed to $P_{min}$, define the output error as the city number that exists in $P_{min}$ but not in $P_{min}'$, and denoted as $P_{min}-P_{min}'$. We define the parameter sensibility as: $$sensibility=\frac{{|P_{min}-P_{min}'|}/{|P_{min}|}}{|A-A'|/{A}}.$$ We plot the minimal matrix approach and graph wavelet approach's sensibility in Figure~\ref{fig:sensibility}. In the minimal matrix approach, when the input parameter error is smaller than 20\%, the output absenteeism group error is less than 5\%. While in the graph wavelet approach, the output absenteeism group error is linear to the input error parameter. This is probably because minimal matrix approach aggregates all the absenteeism score covered by the region, and usually has a much larger city number than the graph wavelet approach, and makes minimal matrix approach better at anti-noise.  All in all, the minimal matrix algorithm focuses on all the cities in the cover group, and has a better global performance at anti-noise, while is inferior to the graph wavelet counterpart in term of running time complexity.
%\begin{figure}[h]
%	\centering
%	\subfigure[matrix]{
%		\includegraphics[width=1.55in,height=1in] {figures/sensibility1.png}
%		\label{fig:sensibility1}
%	}
%	\subfigure[graph]{
%		\includegraphics[width=1.55in,height=1in] {figures/sensibility2.png}
%		\label{fig:sensibility2}
%	}
%	\caption{Sensibility comparison of the two algorithms.}
%	\label{fig:sensibility}
%\end{figure}




% % % % % % % % % % % % % the end% % % % % % % %
%
%\begin{figure}[ht]
%	\centering
%	\subfigure[]{
%		\includegraphics[width=1.55in,height=1in] {figures/Curitiba-Brazil1-cloud.png}
%		\label{fig:holiday}
%	}
%	\subfigure[]{
%		\includegraphics[width=1.55in,height=1in] {figures/pearson1.png}
%		\label{fig:pearson}
%	}
%	\caption{(a) User mobility time series and corresponding absenteeism score, from Dec 5, 2013 to Dec 25, 2013. (b) Pearson correlation score distribution of user mobility and absenteeism score.}
%\label{fig:case3}
%\end{figure}

%Now we show experimental results of our algorithm on highlighted case studies (see Table~\ref{table:list_events}):
%
%\begin{table*}[th] %!htp
%	\renewcommand{\arraystretch}{1}
%	\caption{\label{table:list_events} Selected major events in South America countries}
%	\scriptsize
%	\centering
%	\begin{tabular}{ p{0.5cm}| p{1.5cm} | p{8cm} | p{1.5cm}}
%		\hline
%		\textbf{No.} & \textbf{Date}& \textbf{ Events} & \textbf{Test Areas}   \\ [1ex]
%		\hline
%        1& 2013-06-17 & Brazilian Spring: Protests in over 100 cities, over 2 million people & Brazil  \\		
%        2& 2013-12-02 & Power cut leaves much of Venezuela without electricity & Venezuela \\
%        3& 2013-12-24 & Floods, more than 50,000 people are forced to flee their homes & Brazil\\
%        4& 2013-12-25 & Christmas holiday  & Argentina \\
%        5& 2013-12-30 & Power supply disrupted in heatwave in Buenos Aires, Argentina & Argentina \\
%        6& 2014-04-01 &  M8.2 earthquake struck off the coast of Chile, epicenter is Iquique & Chile  \\
%        7 & 2014-05-21 & Bus strike paralyzes Brazil's biggest city as World Cup looms & Brazil \\			\hline
%	\end{tabular}
%\end{table*}


%The wavelet scales $t_j$ are selected to be logarithmically equispaced between the minimum and maximum scales $t_J$ and $t_1$, with the upper bound $\lambda_{max}$ of the spectrum of $L$. The placement of the maximum scales $t_1$ as well as the scaling function kernel $h$ will be determined by the selection of $\lambda_{min}=\frac{\lambda_{min}}{K}$, where $K$ is a design parameter of the transformation. We then set $t_1$ so that $g(t_1x)$ has power-law decay for $x>\lambda_{min}$, and set $t_J$ so that $g(t_Jx)$ provides monotonicity of the polynomial for $x < \lambda_{max}$. This is achieved by $t_1=\frac{x_2}{\lambda_{min}}$, and $t_J=\frac{x_2}{\lambda_{max}}$. For the scaling function kernel we take $h(x)=\gamma exp(-({{\frac{x}{\lambda_{min}}}})^4)$, where $\gamma$ is set such that $h(0)$ has the same value as the maximum value of $g$.
%
%At each time point the graph has an Z-score vector $f$ and when we get the lowest value for function $< \psi_{t,n},f>$ the corresponding wavelet $\psi_{t,n}$ is reported as an absenteeism pattern.






%\paragraph{\textbf{Event Impact}}
%The absenteeism event impact is defined as:
%\begin{equation}
%AEI = (|W'_f(t_l,n_l;l)|+1)*|W'_f(t_\tau,n_\tau;\tau)|*\frac{t_\tau}{t_l}.
%\end{equation}
%\paragraph{\textbf{Response Time}} The absenteeism event response time is defined as:
%\begin{equation}
% t_{rsp}= \tau-l.
%\end{equation}



% conference papers do not normally have an appendix


% use section* for acknowledgment





% trigger a \newpage just before the given reference
% number - used to balance the columns on the last page
% adjust value as needed - may need to be readjusted if
% the document is modified later
%\IEEEtriggeratref{8}
% The "triggered" command can be changed if desired:
%\IEEEtriggercmd{\enlargethispage{-5in}}

% references section

% can use a bibliography generated by BibTeX as a .bbl file
% BibTeX documentation can be easily obtained at:
% http://mirror.ctan.org/biblio/bibtex/contrib/doc/
% The IEEEtran BibTeX style support page is at:
% http://www.michaelshell.org/tex/ieeetran/bibtex/
%\bibliographystyle{IEEEtran}
% argument is your BibTeX string definitions and bibliography database(s)
%\bibliography{IEEEabrv,../bib/paper}
%
% <OR> manually copy in the resultant .bbl file
% set second argument of \begin to the number of references
% (used to reserve space for the reference number labels box)
%\begin{thebibliography}{1}
%
%\bibitem{IEEEhowto:kopka}
%H.~Kopka and P.~W. Daly, \emph{A Guide to \LaTeX}, 3rd~ed.\hskip 1em plus
%  0.5em minus 0.4em\relax Harlow, England: Addison-Wesley, 1999.
%
%\end{thebibliography}


\bibliographystyle{IEEEtranS}
%\bibliographystyle{splncs03}
\bibliography{my-icdm}



% that's all folks
\end{document}



%
%\begin{figure}[h]
%	\centering
%    {
%		\includegraphics[width= 3in] {figures/performance_compare_bar_graph_brazil.png}
%		\label{fig:distribution2}
%	}
%	\caption{Brazil protest detection performance.}
%	\label{Brazil_performance}
%\end{figure}
%
%
%\begin{figure}[h]
%	\centering
%    {
%		\includegraphics[width= 3in] {figures/performance_compare_bar_graph_mexico.png}
%		\label{fig:distribution2}
%	}
%	\caption{Mexico protest detection performance.}
%	\label{Mexico_performance}
%\end{figure}
%
%
%\begin{figure}[h]
%	\centering
%    {
%		\includegraphics[width= 3in] {figures/performance_compare_bar_graph_venezuela.png}
%		\label{fig:distribution2}
%	}
%	\caption{Venezuela protest detection performance.}
%	\label{Venezuela_performance}
%\end{figure}






% {\textbf{Remarks:}} Essentially, the wavelet frame is generated by kernel function $g(x)$ and scaling function $h(x)$ with $J$ different scales. Those functions are also called filter banks. Figure~\ref{fig:brazil_filter} shows the wavelet filter banks for Brazil graph which we will mention in the experiment part.

% \begin{figure}[h]
% 	\centering
%     {
% 		\includegraphics[width= 3.2in] {figures/brazil_filter.png}
% 		\label{fig:distribution2}
% 	}
% 	\caption{Wavelet filter banks for Brazil graph.}
% 	\label{fig:brazil_filter}
% \end{figure}



%\subsection{Group Absenteeism Event Detection}
%\label{sec:Group Absenteeism Event Detection}
%In our real world, there are various scenarios that causes people silent in social media for a certain amount of time. Take Earthquake emergence for example, when earthquake is happening, people hardly get a chance to use social media ( communication infrastructure might be paralyzed), thus it causes twitter volume in the affected area abnormal low. Some other examples, like flooding, blackout, and so on, often bring about similar phenomena. We call this kind of events as absenteeism event. When absenteeism event is over, and people's circumstance return to normal, it follows some kind of burst in social media since it's natural for people to talk about their experience concerning the absenteeism event. Usually, the severer of the absenteeism event is, the stronger absenteeism of the social media may present during the event, and the stronger burst will follow after.
%
%We propose a novel two-pass absenteeism based event detection algorithm. The underlying rationale of this algorithm is based on the following concepts.
%\begin{enumerate}
%\item At time interval $l$, using algorithm~\ref{algo:event_detection1} to identify absenteeism groups. For each absenteeism group $\mathcal{K}(s_j,a)$, it search burst groups $\mathcal{K}(s_k,b)$ which happens within a time window $L$. We claim that the time delay between $\mathcal{K}(s_j,a)$ and $\mathcal{K}(s_k,b)$ should be smaller than $L$, if the former is caused by the latter. When the time delay is larger that some pre-set $L$,  we claim that the burst group and the absenteeism group are uncorrelated.
%\item $\mathcal{K}(s_j,a)$ and $\mathcal{K}(s_k,a)$ should also be geographically correlated. This is because people are more likely to pay attention to events around where they live. To mathematically measure the geographically closeness of two city sets $\mathcal{K}(s_j,a)$ and $\mathcal{K'}(s_k,b)$, we define closeness $\rho(\mathcal{K}(a),\mathcal{K'}(b))$ as
%\begin{equation}
%\label{eq:eventsimilarity}
%\rho(\mathcal{K}(s_j,a),\mathcal{K'}(s_k,b)) = \frac{|\mathcal{K}(s_j,a)\cap\mathcal{K'}(s_k,b)|}{|\mathcal{K}(s_j,a)|\cdot|\mathcal{K'}(s_k,b)|}
%\end{equation}, where $|\mathcal{K}(\cdot)|$ is the city number. When $\rho$ is above some threshold $\rho_{th}$, we infer that an absenteeism event occurred and that it evolved on social networks into distinct phases: first group absenteeism, followed by a spike or burst in user activity. We denote this absenteeism event as e$\{\mathcal{K}(s_j,a),\mathcal{K}(s_k,b),\tau\}$, $\tau$ is the time delay.
%\item The computational complexity is $O(|\mathcal{I}^{bur}|\cdot |\mathcal{I}^{abs}|\cdot L\cdot |V|)$, and is $O(L|V|^3)$ in worst cases.
%%For instance, taking the power-cut-off for instance, usually only people who live in the affected area will ``yield at " this event a lot because it brings inconvenience to their life. However, people who live outside of the affected areas would hardly mention this event. Thus, to measure the correlation between absenteeism pattern and burst pattern is proposed as:
%\end{enumerate}
%
%\noindent\textbf{Remarks:}
%
%\noindent{In our real world, there are some absenteeism events, for example natural disasters (earthquake, flooding, electricity outage), generates absenteeism group and then burst group in social network. We propose the two-pass algorithm to detect those absenteeism events based on the assumption that those absenteeism and burst groups have strong correlation, both spatially and temporally}.
%
%
%\begin{algorithm}[t]
%\centering
%\captionsetup{font=scriptsize}
%\caption{Two-Pass Absenteeism Event Detection}
%{\footnotesize \begin{algorithmic}[1]
%\STATE {\bf Input:} graph and absenteeism score vector $\mathbf{G}(V,E;f^l)$ at time interval $l$, and time window size $L$.
%\STATE {\bf Output:} absenteeism event set $\mathcal{E}$.	
%\STATE{compute burst group set $\mathcal{I}^{bur}$ by algorithm~\ref{algo:event_detection1}};
%		    	\FORALL {$\tau$ from $l+1$ to $l+L$}
%		    	    \STATE{compute absenteeism group set $\mathcal{I}^{abs}$ by algorithm~\ref{algo:event_detection1}};
%		    	    \FORALL {$\mathcal{K}(s_j,a)\in \mathcal{I}^{bur}$ and $\mathcal{K'}(s_k,b)\in \mathcal{I}^{abs}$}
%				    	    		
%		    	    		    \IF {$\rho(\mathcal{K}(s_j,a),\mathcal{K'}(s_k,b))\ge \rho_{th}$}
%		    	    		    \STATE{add absenteeism event $e\{\mathcal{K}(s_j,a),\mathcal{K}(s_k,b),\tau\}$ to $\mathcal{E}$}
%		    	    	    	\ENDIF
%
%                   \ENDFOR
%
%	            \ENDFOR	
%		
%\RETURN {absenteeism event set $\mathcal{E}$}.
%\end{algorithmic}}
%\label{algo:event_detection}
%\end{algorithm}



